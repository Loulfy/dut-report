\chapter{L'entreprise}
\label{chap:y3s}

\section{Y3S SAS, une ESN polyvalente}

\subsection{Organisation}
La société Y3S est une Société par Actions Simplifiée, fondée il y a
deux ans par ses deux associés, Sylvain \textsc{Leris} et Yann
\textsc{Garras}. Ils rassemblent à eux deux des compétences variées
dans le domaine du WEB (Node.js, PHP, Javascript, AngularJS, JEE),
mobile (iOS, android, cordova), embarqué (Buildroot, kernel unix, ARM,
C/C++, C\#), Système Informatique Géographique (base de données
spatial, QGIS).

Aujourd'hui la société emploie toujours deux personnes :
\begin{itemize}
\item Sylvain \textsc{Leris} : Président
\item Yann \textsc{Garras} : Directeur Général
\end{itemize}

La société fonctionne sur un modèle économique en deux branches
principales :
\begin{itemize}
\item prestation de services : développement d'application WEB, mobile
  ou embarqué pour des clients en fonction de la demande ;
\item éditeur de solution : développement de solution propre à la
  société, comme par exemple une bibliothèque de Système Informatique
  Géographique ;
\item formation : bien qu'elle soit moins présente, c'est une activité
  possible de la société, mais uniquement sur des formation
  spécifique ;
\end{itemize}

Sur tout ce qui est prestation de service et éditeur de solution, Y3S
SAS cherche à obtenir une continuité de maintenance avec leur client.

\subsection{Projets}

La société Y3S SAS travaille en partenariat avec la société ADETEC,
spécialisée dans la conception et la fabrication de systèmes de
télé-transmission, qui sous-traite sa production en Italie. Ce
partenariat leur permet de développer la partie \og Software \fg{} ou
CLOUD, de certain produit d'ADETEC dont le dernier en date fut le
développement d'un kernel UNIX pour une central VoIP sur ARM.

\subsection{Évolution}

La société Y3S SAS souhaiterait évoluer plus dans l'édition de
solution dans le domaine de l'embarqué, domotique ou IoT, qu'il leur
serait propre, comme pourrait l'être la solution d'accès à distance
d'objets connectés. Pour cela il pense renforcer leur partenariat avec
ADETEC, le \og Software \fg{} serait réalisé par Y3S SAS, le \og
Hardware \fg{} par ADETEC et la production en Italie par leur
sous-traitant.

\subsection{Contexte de travail}

La société Y3S SAS a ses bureaux au 16 rue de l'Hermite, Bureau A108,
Bruges. Les outils de utilisés sont :
\begin{itemize}
\item GNU/Linux : différentes distribution, Debian, Raspbian, Ubuntu,
  Xubuntu, pour le développement ou pour les serveurs.
\item CLion : IDE C++ de chez Jetbrain.
\item Visual Studio Code : IDE Javascript de chez Microsoft.
\item Node.js : serveur HTTP en javascript.
\end{itemize}

%%% Local Variables: 
%%% mode: latex
%%% TeX-master: "isae-report-template"
%%% End: 
