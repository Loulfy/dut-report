\chapter*{Conclusion}
\addcontentsline{toc}{chapter}{Conclusion}
\markboth{Conclusion}{Conclusion}
\label{sec:conclusion}

Durant ce stage, j’ai développé le prototype de la solution d'accès distant aux différents services locaux d’un l’objet connecté et plus particulièrement les services WEB liés au protocole HTTP. Pour cela j'ai travaillé sur l'intégralité de la chaîne de développement, j'ai effectué les recherches (pour les choix technologiques), la conception, le développement, la mise en production et bien entendu les tests.

Pour ce projet, j'ai étudié les architectures réseaux des grandes organisations et leurs contraintes ainsi que les différents protocoles nécessaire. J'ai été initié au monde de l'internet des objets, des problèmatiques CLOUD et au monde des microcontrôleur dit industrielle. Les objectifs ont été atteints, toute les fonctionnalités demandés ont été implémenté et testé. Ma solution reste un prototype est devra recevoir une re-mise à niveau pour s'adpater à la solution final qui proposera la société Y3S.

Mon stage m'a permis de m'intégrer dans une entreprise et de m'insérer dans un univers professionnel. J'ai dû acquérir toute les règles de collaboration dans une société. Grâce à mon stage j'ai pu acquérir de solide connaissance en développement réseau sur les sockets natifs, le développement sur plusieurs architectures et l'utilisation de technologies bas niveau en détail avec la programmation sur microcontrôleur et l'implémentation du websocket.

Mon stage m'a conforté dans mon projet professionnel, j'ai souhaite en effet poursuivre mes études pour en apprendre plus sur le développement de technologiques bas niveau, mais orienté dans l'infographie, le son, la vidéo et la 3D.

\chapter*{Webographie}
\addcontentsline{toc}{chapter}{Webographie}
\markboth{Webographie}{Webographie}


\begin{itemize}
\item WebSocket : \\\url{https://developer.mozilla.org/en-US/docs/Web/API/WebSockets_API}\goodbreak\url{/Writing_WebSocket_servers}
\item Node.js : \\\url{https://nodejs.org/en/}
\item Etherws : \\\url{https://pypi.python.org/pypi/etherws/}
\item Node Reverse Wstunnel : \\\url{https://www.npmjs.com/package/node-reverse-wstunnel}
\item Espressif : \\\url{https://espressif.com/en/products/hardware/esp8266ex/overview}
\end{itemize}

%%% Local Variables: 
%%% mode: latex
%%% TeX-master: "isae-report-template"
%%% End: 

